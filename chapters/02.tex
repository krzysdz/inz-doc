\chapter{Problem analysis}

Traditional capture the flag game requires the player to take a flag or a similar object from opponent base and bring it back to their own. Similarly, in cybersecurity CTF the objective is obtaining a flag from the \textit{opponent}. There are two formats of CTFs - attack-defence and jeopardy. In the first type competing teams have to both attack other teams and protect their own system. The latter requires participants to \textit{steal} the flags from organiser-prepared challenges. The tools developed for this thesis are supposed to help programmers understand security issues, therefore only jeopardy-style are considered.

\section{Literature research}

Using CTF-style challenges for cybersecurity education is not a new idea. Gamification is helpful for increasing user engagement and works in cybersecurity too \cite{bib:exploring-game-design}. An analysis of PicoCTF 2013 competition by Chapman et al. \cite{bib:picoCTF} shows that properly prepared CTFs are suitable even for high school students with no security experience. According to the survey answers the teachers observed that the students put more effort into the CTF than usually during lessons.

\subsection{CTF as university laboratories}

CTF challenges are already used during cybersecurity laboratories on some universities. The topic has been approached in various ways by the instructors.

Karagiannis and Magkos \cite{bib:Karagiannis2021} use CTFd for managing the challenges. Challenges were divided into subsequent steps to guide the participants through the process and present the knowledge better. \href{https://vulnhub.com/}{VulnHub} VMs were used as the challenges and most of them were running locally on students' computers.

In \cite{bib:teaching-ctf-PL} Ksiezopolski et al. present their web application security course based on CTF. Students connect to the infrastructure using individual VPN configurations. A challenge must be started by the student using a web portal to make it available on the VPN network. This solution allows \textit{cheap} scaling by starting challenges as required.

\subsection{Technical details of CTF organization}

CERT Polska describes their approach to hosting CTF competitions in \cite{bib:hack.cert.pl}. Their solution for hosting \textit{web} and \textit{pwn} challenges is based on Docker Compose. This decision allows them using different environments for each challenge, which should behave identically on different devices, therefore eliminating compatibility problems. Routing network traffic to appropriate containers is achieved by using server blocks with \mintinline{nginx}|server_name| and \mintinline{nginx}|proxy_pass| directives in nginx configuration files.

Authors detail also their remarks regarding task design. The article suggests using NsJail to prevent accidental or purposeful denial of service (DOS) on high-risk challenges such as those leading to RCE. Another helpful idea is using \href{https://github.com/Supervisor/supervisor}{supervisord} for managing multiple processes inside a single container.

\section{Existing solutions}
\label{sec:existing-solutions}

There are many open-source platforms developed for hosting CTF competitions and some of them were created with education in mind. This section shortly describes some of the most popular solutions which could be used to introduce CTF challenges in the context of education. However, the presented projects frequently solve only a part of the problem - either providing the user interface or the infrastructure for hosting challenges.

\subsection{CyTrONE}

\href{https://github.com/crond-jaist/cytrone}{CyTrONE} \cite{bib:cytrone} is a cybersecurity training framework developed at the Japan Advanced Institute of Science and Technology. The system operates on virtual machines hosting the vulnerable applications which are created based on YAML configuration files. The VMs are cloned for each student requesting a \textit{cyber range}. Guest VMs may include a desktop with tools required for exploitation, so that users can just connect to the VM and use the tools available there. Besides \textit{cyber ranges}, the system supports also text and multiple-choice quizzes. Administrators (called \textit{training coordinators}) manage the database of \textit{cyber ranges} and quizzes using a web UI or iOS application. Students have access to the training through the Moodle learning management system which CyTrONE closely integrates with.

\subsection{kCTF}

\href{https://github.com/google/kctf}{kCTF} is a Kubernetes-based infrastructure for hosting CTFs developed by Google. Thanks to the use of Kubernetes the challenges should scale automatically as required. The templates available for kCTF include healthcheck containers and use NsJail for sandboxing the incoming connections and securing challenges with remote code execution. kCTF provides only the infrastructure, so the user interface must be deployed separately.

\subsection{CTFd}
\label{ssec:CTFd}

\href{https://github.com/CTFd/CTFd}{CTFd} is a highly configurable CTF framework. Challenges can be configured to depend on other challenges so that they are solved in order. It supports plugins which extend the functionality. It is liked for its clear UI which is intuitive even for beginners \cite{bib:CTF-analysis, bib:bangladesh-CTFd}. Unfortunately, the official multiple choice challenge plugin is available only in paid versions. The targets for web challenges can only be linked to and must be hosted separately.

\subsection{FBCTF}
\label{ssec:FBCTF}

\href{https://github.com/facebookarchive/fbctf}{Facebook CTF (FBCTF)} is a \textbf{deprecated} platform for hosting CTF competitions. Because it targets mostly competitions it has a highly-configurable scoring system. Besides jeopardy-style challenges it supports also quizzes (text answer) and \textit{King of the hill} games in which teams compete for control over a target system. The \textit{flag levels}, which are the jeopardy tasks, have a file or a link to the target attached, but how the target system is maintained is left up to the administrator. According to Karagiannis et al. \cite{bib:CTF-analysis} FBCTF is best suited for organizing CTF competitions as events.

\subsection{Mellivora}

\href{https://github.com/Nakiami/mellivora}{Mellivora} is a very lightweight and simple CTF engine. It allows \textit{challenge chains} where a task becomes available only after a specific other challenge has been solved. Challenges have a BBCode description and can contain files, but if a web service is required it must be hosted separately and linked in the description. Answers must be text and can be marked automatically or manually by the administrators.

\subsection{Root the Box}

\href{https://github.com/moloch--/RootTheBox}{Root the box} is a CTF scoring engine with many configurable features. It supports multiple flag types including text (static and regular expressions), datetime, multiple choice and file which may be really useful when used for educational purposes. Support for the \href{https://github.com/juice-shop/juice-shop-ctf}{Juice Shop CTF CLI} makes it easy to add challenges from that CTF, however this feature does not differentiate it from \hyperref[ssec:CTFd]{CTFd} and \hyperref[ssec:FBCTF]{FBCTF}, which are also supported. However, Karagiannis et al. \cite{bib:CTF-analysis} claims that Root the Box is too complex for beginners, therefore not a good solution for education.
