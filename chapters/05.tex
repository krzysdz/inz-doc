\chapter{Internal specification}

% \section{Concept of the system}
% Let's just skip this section. I believe that the requirements contain enough information

\section{System architecture}

The system is managed by a server running on Node.js. As presented on Fig. \ref{fig:system-architecture}, this process manages the MongoDB database, challenge containers via Docker Engine API and the nginx proxy configuration. Challenges are run as Docker containers to enable easy configuration by just puling an appropriate image and each of them is bound to a different subdomain of a dedicated challenge domain thanks to the single proxy server.

\begin{figure}
    \centering
    % Text renders wrong without inkscapelatex=false
    \includesvg[inkscapelatex=false,scale=0.785]{img/system-architecture.svg}
    \caption{Visual representation of system architecture.}
    \label{fig:system-architecture}
\end{figure}

The nginx process is the only one exposed publicly. Other software can listen only on localhost for security reasons.

\section{Database structure}

The use of a NoSQL database, allowed for a more flexible schema, based on the desired access to the data. A mix of embedded data model and normalized data model is used. The used database schema is presented on Fig. \ref{fig:db-schema}. Despite preferring the denormalized model, it was necessary to keep some relations between documents. This is achieved by storing fields with \texttt{\_id}s of referenced documents. There is also a relation between keys of embedded documents in the \texttt{users} collection, which are stored as stringified \texttt{ObjectId}s and documents from \texttt{tasks} and \texttt{challenges}. This relation, however, is not used in queries to the database, but managed completely by the server code.

There are two additional indexes:

\begin{itemize}
    \item a unique index on the \texttt{subdomain} field in the \texttt{challenges} collection to make sure that there are no duplicate subdomains,
    \item a unique index with a collation on the \texttt{name} field in the \texttt{categories} collection to accelerate case-insensitive search by category name and prevent duplicates.
\end{itemize}

The view \texttt{fullTasks} is a view collection on the \texttt{tasks}, which uses an aggregation pipeline to embed documents from the \texttt{challenges} collection using \texttt{\$lookup} where necessary. The pipeline is presented on Fig. \ref{fig:fullTasks-agg}.

\begin{figure}
    \centering
    % inkscapelatex=false, because the generated LaTeX file is invalid
    \includesvg[inkscapelatex=false,scale=0.87]{img/db-schema.svg}
    \caption{Visual diagram of the database schema.}
    \label{fig:db-schema}
\end{figure}

\section{Code organization}

The code is divided into ECMAScript modules. The main one is \texttt{index.js}, which imports all the other required modules, some external tools, configures and starts the server. Each module serves a separate function.

\subsection{Middleware}

Two modules are responsible for middleware functions. Middleware is a function called during request processing, which can act on the request and response objects and pass execution to functions declared later in the router stack.

One of them, \texttt{src/middleware.js}, contains general utility middleware functions:

\begin{itemize}
    \item \texttt{authenticated} - a middleware which returns 401 if the user is not logged in,
    \item \texttt{adminOnly} - a middleware which returns 403 if the user is not an administrator,
    \item \texttt{addCategoriesList} - a middleware executed before any routers, which fetches all category names and makes them available as response locals to use within templates.
\end{itemize}

The other module was separated, because it alters the request object and has an associated \texttt{.d.ts} typings file to help with type checking and code suggestions. It is \texttt{src/flash.js} and adds the following methods to each request object:
\begin{itemize}
    \item \mintinline{js}|flash(message, category = "info")| - a function which adds a message with category to a list of flash messages connected with the session
    \item \texttt{getFlashedMessages(options)} - a function which returns the flashed messages with associated categories. This function can also filter the returned flashes by category. It is also available in response locals for direct use in templates.
\end{itemize}

The behaviour of this middleware is supposed to mimic the message flashing \href{https://flask.palletsprojects.com/en/2.2.x/quickstart/#message-flashing}{functionality} of the Flask framework.

\subsection{Routers}

The \texttt{index.js} module provides only the request handler for the root path \texttt{/}. Other paths are delegated to separate routers, which live inside modules under \texttt{src/routes}. The following paths are registered:
\begin{itemize}
    \item \texttt{/auth} handled by \texttt{authRouter} from the \texttt{auth.js} module. It serves the register, login and logout pages, as well as handles password change requests.
    \item \texttt{/profile} guarded by the \texttt{authenticated} middleware, handled by \texttt{profileRouter} from the \texttt{profile.js} module. It serves only the profile page.
    \item \texttt{/admin} guarded by the \texttt{authenticated} middleware, handled by \texttt{adminRouter} from the \texttt{admin.js} module.\\
    The router uses additional middleware \texttt{adminOnly} and \texttt{express.json()}. It renders only a single page - the admin panel. It provides a REST-like JSON API, which is used by a script on the admin panel page.
    \item \texttt{/category} handled by \texttt{categoryRouter} from the \texttt{category.js} module. It serves category pages with lists of related tasks and a category list page \texttt{/categories/}.
    \item \texttt{/task} handled by \texttt{taskRouter} from the \texttt{task} module. It serves task details pages and handles challenge and quiz submissions. All submissions are guarded with \texttt{authenticated}. The details page is handled separately for anonymous and other users as described in section \ref{chap:types-of-users}.
\end{itemize}

\subsection{Markdown parser}

To parse descriptions in Markdown the \href{https://marked.js.org/}{\texttt{marked}} parser is used together with syntax highlighting library \href{https://highlightjs.org/}{\texttt{highlight.js}}. Because the processing can take some time and the execution runtime is single-threaded, it is offloaded to a \href{https://nodejs.org/dist/latest-v19.x/docs/api/worker_threads.html}{worker thread}. The thread is terminated after a timeout passes to avoid blocking the thread pool. \cite{bib:event-loop-explained,bib:event-loop-blocking}

The first module responsible for this functionality \texttt{src/markedThread.js} listens to messages on \texttt{parentPort} and responds to them with an HTML string containing parsed Markdown.

The second module \texttt{src/markdown.js} exports a \texttt{processMarkdown} function, which manages the worker thread. It returns a \href{https://tc39.es/ecma262/multipage/control-abstraction-objects.html#sec-promise-objects}{\texttt{Promise}}, which is resolved after the worker sends the rendered string. If the worker emits an error event, the promise is rejected. After a specified time passes the worker is terminated and the promise rejected, unless it had finished successfully earlier.

\subsection{Database connection}

The module \texttt{src/db.js} does not export any functions. When it is imported it creates a \texttt{MongoClient} connected to the database server and opens the database specified in the configuration file. It calls then an \texttt{async} function \texttt{setupDb}, which prepares necessary indexes and views. Both the client (\texttt{client}) and database (\texttt{db}) are exported. All modules importing either exported variable reuse the same client connection.

\subsection{Challenge management}

Challenges are managed using functions from the \texttt{src/challenges.js} module. The module has the following functions:
\begin{itemize}
    \item \mintinline{js}|getChallengeContainers(all = false)| - returns challenge containers (only running, unless \texttt{all} is \texttt{true}). It relies on label \texttt{pl.krzysdz.inz.challenge-kind} being present.
    \item \texttt{startupChallenges()} - removes all nginx subdomain configuration files, stops and removes all challenge containers and calls \texttt{addChallenge} with each challenge document from the database.
    \item \mintinline{js}|addChallenge(challengeDoc, reload = false)| - verifies the image tag, then calls functions \texttt{startChallengeContainer} and \texttt{createNginxConfig}. If \texttt{reload} is \texttt{true}, calls a function responsible for reloading nginx configuration.
    \item \texttt{checkTag(challengeDoc)} - checks if the document contains an image with a tag. If there is no tag, \texttt{:latest} is appended to the image name. Returns the \texttt{challengeDoc} with corrected \texttt{taskImage} field.
    \item \texttt{startChallengeContainer(challengeDoc)} - pulls the image specified in the argument, stops and removes containers with the same name if they exist, creates a new container and starts it. If \texttt{resetInterval} is not 0, \texttt{setInterval} is created, which uses \texttt{restartChallengeContainer} to periodically restart the container.
    \item \texttt{restartChallengeContainer(port, challengeDoc)} - restarts the container specified in \texttt{challengeDoc}. This function is not exported and called only in interval, which is created after starting the container for the first time. It stops and removes the container including volumes, creates a new container reusing the same port and starts the new container.
    \item \texttt{createNginxConfig(subdomain, port)} - creates an nginx configuration file in \texttt{nginx/conf/subdomains} directory, which defines an HTTPS server proxying request made to the subdomain of domain specified in config to the specified port on localhost.
    \item \texttt{reloadNginx()} - executes the nginx binary with \texttt{-s reload} and prefix set to \texttt{./nginx} from the project directory to instruct the proxy to reload its configuration.
\end{itemize}

\section{Sequence diagrams}

Basic system functionality in a simplified form can be presented in form of UML sequence diagrams. Figures \ref{fig:seq-register}, \ref{fig:seq-login} and \ref{fig:seq-change-pass} demonstrate flows related to authentication and account management. The next three diagrams \ref{fig:seq-task}, \ref{fig:seq-task-flag} and \ref{fig:seq-task-answer} present operations connected with tasks. The diagrams are simplified, to avoid repeating the process of fetching task and checking user progress before rendering the page.

% Text in diagrams doesn't look good without inkscapelatex=false
% Too wide or positioned incorrectly

\begin{figure}
    \centering
    \includesvg[inkscapelatex=false]{uml/render/seq-register.svg}
    \caption{Registration sequence diagram}
    \label{fig:seq-register}
\end{figure}

\begin{figure}
    \centering
    \includesvg[inkscapelatex=false]{uml/render/seq-login.svg}
    \caption{Logging in sequence diagram}
    \label{fig:seq-login}
\end{figure}

\begin{figure}
    \centering
    \includesvg[inkscapelatex=false]{uml/render/seq-change-pass.svg}
    \caption{Password change sequence diagram}
    \label{fig:seq-change-pass}
\end{figure}

\begin{figure}
    \centering
    \includesvg[inkscapelatex=false, scale=0.9]{uml/render/seq-open-solve-task.svg}
    \caption{Navigation and solving a task}
    \label{fig:seq-task}
\end{figure}

\begin{figure}
    \centering
    \includesvg[inkscapelatex=false]{uml/render/seq-task-flag.svg}
    \caption{Submitting challenge flag}
    \label{fig:seq-task-flag}
\end{figure}

\begin{figure}
    % Don't use inkscapelatex=false to render \ref, make font smaller manually
    \fontsize{10}{12}\selectfont
    \centering
    \includesvg[]{uml/render/seq-task-answer.svg}
    \caption{Answering task question}
    \label{fig:seq-task-answer}
\end{figure}

Actions available for the administrators involve an additional participant - a script running in the browser which is responsible for managing the UI and communication with the server. Because the panel is not rendered on the server, the process of loading it is more complicated as can be seen on Fig. \ref{fig:seq-admin-load}. The most complicated operation is adding a new task. The flow is presented on \ref{fig:seq-admin-add-task} and requires interactions with even more system components.

\begin{figure}
    \centering
    \includesvg[inkscapelatex=false, scale=0.9]{uml/render/seq-admin-load.svg}
    \caption{Loading administrator panel}
    \label{fig:seq-admin-load}
\end{figure}

\begin{figure}
    \centering
    \includesvg[inkscapelatex=false, scale=0.57]{uml/render/seq-admin-add-task.svg}
    \caption{Adding a new task}
    \label{fig:seq-admin-add-task}
\end{figure}
