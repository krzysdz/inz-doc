\chapter{Requirements and tools}

\begin{itemize}
\item functional and nonfunctional requirements
\item use cases (UML diagrams)
\item description of tools
\item methodology of design and implementation
\end{itemize}

\section{Functional requirements}

Functional requirements list functionality available in the system. Each of the requirements contains a description detailing the desired behaviour.

\begin{enumerate}
	\item \textbf{Account creation}: Users must be able to register an account in the system.
	This operation requires username and password submission from an HTML form in a POST request.
	Registration is allowed only using unique username. If there already exists an account in the system with the same username, the action must be refused.
	As a result of a successful registration, a document with the username, cryptographic hash of the user's password and a default role \texttt{"user"} is inserted into a collection storing user accounts. After the registration succeeds, the user is redirected to the login page.

	\item \textbf{Signing in}: Users must be able to log into their accounts.
	Username and password must be sent in a POST request as HTML form data. The operation must fail if there is no account in the database with the provided username or if the password hash does not match the one stored in the database.
	If there has been no failure, a session is created and the user is redirected to the home page.

	\item \textbf{Logging out}: Users must be able to log out of their accounts.
	It is expected that the user is signed in when they log out. This operation destroys the session (if any) and redirects to the home page.

	\item \textbf{Changing password}: Users must be able to change their account password.
	New password must be sent in a POST request as HTML form data. User must be logged in in order to change the password.
	As a result of this operation, user's password hash in the database is updated.

	\item \textbf{Listing categories}: Users must be able to see a list of categories that exist in the system. List of categories must link to category pages.

	\item \textbf{Displaying category}: Users must be able to see category details on a category page. The details should include category name, description and a list of related tasks.

	\item  \textbf{Displaying task}: Users must be able to display task details on task pages. The details should include task name, description, challenge URL, hints (if any), and if the user is logged in, a flag submission form. If the task has been solved by the user, instead of the flag submission form, a question is displayed.

	\item \textbf{Solving challenge}: Users must be able to submit a form with flag from the task page. This action is allowed only for signed in users. Submitted flag must be verified with the one stored in the database for the given challenge. If the flags match, a success message should be shown to the user and date of solving the challenge by user ought to be saved to the database. Otherwise, a message informing the user about incorrect flag should be presented.

	\item \textbf{Answering quiz}: User who have solved the main challenge from a task, should be able to to see a question on the task page and be able to answer it. Only signed in users can submit answers. Checkboxes which status indicates whether the user thinks that they are correct must be presented along answers from the database. After the answers are submitted, the quiz should be disabled without a button to submit, checkboxes disabled and reflecting the user's answer and an indication which answers were correct.

	\item \textbf{Administrator panel}: Users with the \texttt{"admin"} role (administrators) should have access to a separate administration panel. The panel should expose additional functionality related to the system management.

	\item \textbf{Listing users}: Users with access to the administrator panel should be able to get a list of user account in the system. Each entry must contain information about username and user role. The list should be paginated as there may exist many accounts.

	\item \textbf{Changing user permissions}: Administrators should be able to grant the \texttt{"admin"} role to other users, as well as change it back to \texttt{"user"}.

	\item \textbf{Deleting user}: Administrators should be able to remove user accounts from the database. It must not be possible to remove own user account this way.

	\item \textbf{Creating category}: Administrators must be able to create new categories. The categories must have a name and description. The description must support Markdown input.

	\item \textbf{Editing category}: Administrators must be able to edit the name and description of existing categories.

	\item \textbf{Creating task}: Administrators must be able to add new tasks to the system. For each task it must be possible to set the name, description (in Markdown), hints, challenge details, question, answers to the question. Each answer must be marked as correct or incorrect.\\
	Challenge details must include:
	\begin{itemize}
		\item Docker image to pull and start,
		\item subdomain used for serving the challenge,
		\item flag that the users will try to find,
		\item an interval specifying how frequently the challenge container should be regenerated
	\end{itemize}

	\item \textbf{Starting challenges}: The system must be able to pull challenge images, create and containers and direct connections to specified subdomains into appropriate containers. This must be done automatically during system startup and for each new challenge added when creating new tasks.
\end{enumerate}

\section{Non-functional requirements}

\begin{enumerate}
	\item \textbf{Responsiveness}: UI should properly scale across different display sizes. It must be mobile-friendly.

	\item \textbf{Accessibility}: There should be no errors in the Accessibility section of a \href{https://webhint.io/}{webhint} scan.

	\item \textbf{Visual consistency}: A single set of styling rules, such as colours, fonts and icons should be used across whole user interface.

	\item \textbf{Page load performance}: The system should have a score of over 90 in \href{https://pagespeed.web.dev}{PageSpeed Insights} report for mobile.

	\item \textbf{Compatibility}: User interface should work in latest (as of January 2023) versions of Firefox, Firefox ESR, Chrome and Safari browsers for desktops and mobile devices. Basic system functionality, except for the administrator panel, should be available in browsers with JavaScript disabled.
\end{enumerate}

\section{Use cases}

The use case diagram presented on Fig. \ref{fig:use-case-diag} shows actions offered to actors using the system. There are two actors, with different sets of allowed interactions. The actor \textit{User} represents anyone with access to the system. Users with special account role \texttt{"admin"}, which allows them to perform operations related to management of the system.

\begin{figure}
	\centering
	\includesvg[scale=0.9]{uml/render/usecase.svg}
	\caption{Use case diagram}
	\label{fig:use-case-diag}
\end{figure}

\section{Tools}

The implementation of the project significantly benefited from publicly available tools. Used tools are divided into two classes, depending on the way they were used.

\subsection{Core tools}

The system is built on tools, which are regarded to as \textit{core tools}. These are required for operation and are a part of the system architecture.

\subsubsection{Node.js + Express + EJS}

\subsubsection{MongoDB}

\subsubsection{Docker}

\subsubsection{nginx}

\subsubsection{Bootstrap}

\subsection{Development tools}

The following tools are in no way required for the systems. These were used to aid development.

\subsubsection{Visual Studio Code}

Visual Studio Code is a multi-platform code editor that was heavily used for writing the software. It was chosen for multiple reasons, most important of which was familiarity and experience with the tool. This editor supports many languages, especially for JavaScript and related web technologies. Particularly notable is built-in TypeScript support, which can be used with JSDoc comments to improve IntelliSense suggestions. A huge advantage of Visual Studio Code is a broad selection of available extensions.

\subsubsection{Git}

The project is stored in a Git repository to track changes in the code. The repository is synchronised with GitHub to share it between devices.

\subsubsection{ESLint}

ESLint is a JavaScript linter, which can be used to detect problems and potential issues in code. It was used with a Visual Studio Plugin, which automatically analysed open files and provided in-editor warnings and suggestions.

\subsubsection{Prettier}

Prettier is a popular code formatting tool. It was used to maintain a consistent style in JavaScript files. Thanks to Visual Studio Code plugin the tool could be used as a default formatter in the editor. A plugin for ESLint allowed marking improperly formatted code as lint error.
