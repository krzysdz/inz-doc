\chapter{Requirements and tools}

\begin{itemize}
\item functional and nonfunctional requirements
\item use cases (UML diagrams)
\item description of tools
\item methodology of design and implementation
\end{itemize}

\section{Functional requirements}

Functional requirements list functionality available in the system. Each of the requirements contains a description detailing the desired behaviour.

\begin{itemize}
	\item \textbf{Account creation}: Users must be able to register an account in the system.
	This operation requires username and password submission from an HTML form in a POST request.
	Registration is allowed only using unique username. If there already exists an account in the system with the same username, the action must be refused.
	As a result of a successful registration, a document with the username, cryptographic hash of the user's password and a default role \texttt{"user"} is inserted into a collection storing user accounts. After the registration succeeds, the user is redirected to the login page.

	\item \textbf{Signing in}: Users must be able to log into their accounts.
	Username and password must be sent in a POST request as HTML form data. The operation must fail if there is no account in the database with the provided username or if the password hash does not match the one stored in the database.
	If there has been no failure, a session is created and the user is redirected to the home page.

	\item \textbf{Logging out}: Users must be able to log out of their accounts.
	It is expected that the user is signed in when they log out. This operation destroys the session (if any) and redirects to the home page.

	\item \textbf{Changing password}: Users must be able to change their account password.
	New password must be sent in a POST request as HTML form data. User must be logged in in order to change the password.
	As a result of this operation, user's password hash in the database is updated.

	\item \textbf{Listing categories}: Users must be able to see a list of categories that exist in the system. List of categories must link to category pages.

	\item \textbf{Displaying category}: Users must be able to see category details on a category page. The details should include category name, description and a list of related tasks.
\end{itemize}

% TODO
\subsection{Displaying task}
\subsection{Solving challenge}
\subsection{Answering quiz}
\subsection{Administrator panel}
\subsection{Listing users}
\subsection{Changing user permissions}
\subsection{Deleting user}
\subsection{Creating category}
\subsection{Editing category}
\subsection{Adding tasks}
\subsection{Starting challenges}

\section{Non-functional requirements}

\subsection{Responsiveness}

UI should properly scale across different display sizes. It must be mobile-friendly.

\subsection{Accessibility}

There should be no errors in the Accessibility section of a \href{https://webhint.io/}{webhint} scan.

\subsection{Visual consistency}

A single set of styling rules, such as colours, fonts and icons should be used across whole user interface.

\subsection{Page load performance}

The system should have a score of over 90 in \href{https://pagespeed.web.dev}{PageSpeed Insights} report for mobile.

\subsection{Compatibility}

User interface should work in latest (as of January 2023) versions of Firefox, Chrome and Safari browsers for desktops and mobile devices. Basic system functionality, except for the administrator panel, should be available in browsers with JavaScript disabled.

\section{Use cases}

The use case diagram presented on Fig. \ref{fig:use-case-diag} shows actions offered to actors using the system. There are two actors, with different sets of allowed interactions. The actor \textit{User} represents anyone with access to the system. Users with special account role \texttt{"admin"}, which allows them to perform operations related to management of the system.

\begin{figure}
	\centering
	\includesvg[scale=0.9]{uml/render/usecase.svg}
	\caption{Use case diagram}
	\label{fig:use-case-diag}
\end{figure}

\section{Tools}



\subsection{Node.js + Express}

\subsection{MongoDB}

\subsection{Docker}

\subsection{nginx}

\subsection{Bootstrap}
