\chapter{Conclusions}

\section{Results}

The project at least partially meets all the requirements listed in chapter \ref{chap:req-and-tools}. The only requirement not met completely is compatibility, but it should not cause problems for the overwhelming majority of users.

\section{Future development ideas}

Although the project is usable and demonstrates a general idea there is still an area for improvements. This section lists some propositions for future project development.

\subsection{More content}

The most valuable part of the system for regular users is the content - description of vulnerabilities and tasks. The currently offered examples just scratch the surface of web application security. Expanding the topics, adding more diverse tasks and introducing new categories certainly would enrich the user experience. Broadening the repertoire could make the service useful to a wider audience as well as present the vulnerabilities in more details.

Increasing the number of tasks in a single category with different difficulty levels creates a risk of a less legible and usable category pages and could discourage users less experienced with those categories. To avoid that problem the tasks could be tagged with a difficulty level and ordered by it, so everyone will be able to choose what best matches their abilities and ambitions.

\subsection{Better task management}

Current task management is restricted to task creation, which is rather complicated and involves filling a complex form properly in one try. This aspect can be greatly improved by introducing the following changes:
\begin{itemize}
	\item allow editing existing tasks - ideally the edits would be applied separately to each part (name, description, question, challenge, etc.) of the task to reduce the complexity of changes and make the implementation simpler; this functionality could be used to fix mistakes in the text or update challenges if the image is fixed to a specific version,
	\item task drafts - tasks hidden from users, which may not have all the required fields filled,
	\item health checks and container status - monitoring the status of a challenge container and periodical checks of the application inside could help diagnose issues and prevent users accidentally or purposefully taking the tasks down; could be paired with an alert system to administrators and automated restarts,
	\item forced manual restart - in case a task is down or vandalized and the automatic reset is not going to happen soon, the administrators should be able to force a challenge container reset,
	\item time to the next restart - when the challenge will be recreated from the initial state automatically, this could be also shown on the task page to users.
\end{itemize}

\subsection{User overview for administrators}

To help with debugging possible issues user actions such as flag and answer submission could be logged, even if the flag is invalid. An overview of such events could be then checked by an administrator to verify the user findings, detect misconfigured challenges or hint the user if e.g. they found a bait flag left by another user.

\subsection{Progress tracking}

The users can see their progress for a single task when they open it. A progress bar or a colour change could be added to task tiles on category pages to indicate whether a task has been solved and to what level (challenge solved, task answered, all challenges solved).

\subsection{Increasing user engagement}

User engagement could be improved by introducing a public scoreboard and therefore an element of competitiveness. However, publishing own results should not be mandatory not to unnecessarily put pressure on privacy-focused or less confident users.
