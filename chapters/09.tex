\chapter{Listings}

\begin{figure}
	\centering
	\begin{minted}[linenos, breaklines, frame=lines]{yaml}
# mongod.conf

# for documentation of all options, see:
#   http://docs.mongodb.org/manual/reference/configuration-options/

# Where and how to store data.
storage:
  dbPath: /var/lib/mongodb

# where to write logging data.
systemLog:
  destination: file
  logAppend: true
  path: /var/log/mongodb/mongod.log

# network interfaces
net:
  port: 27017
  bindIp: 127.0.0.1

# how the process runs
processManagement:
  timeZoneInfo: /usr/share/zoneinfo

replication:
  replSetName: rs0
	\end{minted}
	\caption{Example \texttt{/etc/mongod.conf} configuration file. Based on the default \href{https://github.com/mongodb/mongo/blob/e4fff3e1fe7b31b25cedde7b05205325b47b4a7d/debian/mongod.conf}{config} for Debian and Ubuntu.}
	\label{fig:example-mongod}
\end{figure}

\begin{figure}
	\centering
	\begin{minted}[linenos, breaklines, frame=lines]{INI}
# /etc/systemd/system/inz-nginx.service

[Unit]
Description=nginx proxy for inz
Documentation=https://nginx.org/en/docs/
After=network-online.target remote-fs.target nss-lookup.target
Wants=network-online.target

[Service]
Type=forking
PIDFile=/run/nginx.pid
WorkingDirectory=/inz
ExecStart=/usr/sbin/nginx -p /inz/nginx -c /inz/nginx/conf/nginx.conf
ExecReload=/bin/sh -c "/bin/kill -s HUP $(/bin/cat /run/nginx.pid)"
ExecStop=/bin/sh -c "/bin/kill -s TERM $(/bin/cat /run/nginx.pid)"

[Install]
WantedBy=multi-user.target
	\end{minted}
	\caption{Example unit file for the nginx proxy. Project root is assumed to be \texttt{/inz}.}
	\label{fig:example-nginx-service}
\end{figure}

\begin{figure}
	\centering
	\begin{minted}[linenos, breaklines, frame=lines]{INI}
# /etc/systemd/system/inz.service

[Unit]
Description=inz server
After=network-online.target remote-fs.target nss-lookup.target mongod.service inz-nginx.service
Wants=network-online.target
Requires=mongod.service inz-nginx.service

[Service]
Type=simple
WorkingDirectory=/inz
Environment="NODE_ENV=production"
# Remember to change the secret!
Environment="SECRET=SECRET"
ExecStart=/usr/bin/node /inz/index.js
ExecStop=/bin/kill -s TERM $MAINPID

[Install]
WantedBy=multi-user.target
	\end{minted}
	\caption{Example unit file for the main server. Project root is assumed to be \texttt{/inz}.}
	\label{fig:example-server-service}
\end{figure}

\begin{figure}
	\centering
	\begin{minted}[linenos, breaklines, frame=lines]{js}
import { MongoClient } from "mongodb";
import { DB_NAME, DB_URL } from "./config.js";

const USERNAME = "administrator_username";

const client = new MongoClient(DB_URL, { directConnection: true });
await client
	.db(DB_NAME)
	.collection("users")
	.updateOne({ _id: USERNAME }, { $set: { role: "user" } });
await client.close();
	\end{minted}
	\caption{A simple Node.js script which makes the user \texttt{administrator\_username} an administrator.}
	\label{fig:make-admin-script}
\end{figure}

\begin{figure}
	\centering
	\begin{minted}[linenos, breaklines, frame=lines, tabsize=4, obeytabs]{js}
[
	// add challenge as "challenge" array of 1 document
	{ $lookup: {
			from: "challenges",
			localField: "challengeId",
			foreignField: "_id",
			as: "challenge",
	}},
	// add challenges from answers as "answerChallenges" array
	{ $lookup: {
			from: "challenges",
			localField: "answers.challengeId",
			foreignField: "_id",
			as: "answerChallenges",
	}},
	// select and modify returned fields
	{ $project: {
			name: 1,
			categoryId: 1,
			descriptionMd: 1,
			descriptionRendered: 1,
			question: 1,
			hints: 1,
			visible: 1,
			// Replace the "challenge" array with the only document inside it
			challenge: { $first: "$challenge" },
			// Add a challenge field (subdocument) to each answer with challengeId
			answers: {
				$map: {
					input: "$answers",
					as: "a",
					in: {
						$mergeObjects: [
							"$$a",
							{ challenge: {
									$first: {
										$filter: {
											input: "$answerChallenges",
											cond: { $eq: ["$$this._id", "$$a.challengeId"] },
											limit: 1,
}}}}]}}}}}]
	\end{minted}
	\caption{Aggregation pipeline for the \texttt{fullTasks} collection.}
	\label{fig:fullTasks-agg}
\end{figure}
