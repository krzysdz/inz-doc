\chapter{Introduction}

Latest reports show that on average people spend over 6.5 hours daily using the internet \cite{bib:digital-2023}. Computers and internet play a huge role in today's world. Unfortunately many services and applications are not secure. Consequences of vulnerabilities may be very different and vary from almost negligible to extremely severe. Some examples include service inaccessibility, data leaks and network takeovers. Sometimes seemingly harmless issues may be exploited in a series of attacks and cause much bigger problems. Like in many areas the best solution to these issues is prevention. Although code review and testing may help with problem detection and fixing, it would be much better if these issues were not created at the first place. According to Michał Sajdak the source of many vulnerabilities and one of the most important problems in security is the lack of knowledge about security issues \cite{bib:securitum-wstep}. A developer, especially if not experienced, may not be aware of potential issues with their code. Education about vulnerabilities may help programmers look on their code from a different perspective therefore making it more secure.

The aim of the thesis is preparing a system useful for teaching cybersecurity. The system is intended to utilize Capture The Flag (CTF) challenges for demonstrating the vulnerabilities. This is a popular challenge type requiring the user to exploit a security issue and find a hidden flag.

The thesis consists of the planning and preparation of the project, creation of architecture capable of running challenges, designing and implementing user interface and a management panel. Additionally, existing solutions and ideas related to the problem were analysed before planning the system. The following chapters are included in this thesis:

\begin{itemize}
	\item \textbf{Problem analysis} - analysis of the problem and available literature, description of existing solutions,
	\item \textbf{Requirements and tools} - formulation of functional and non-functional requirements, use case analysis and selection of third-party tools and technologies, the design and planning process
	\item \textbf{External specification} - specification of hardware and software requirements, installation and setup process, user and administrator manuals, security analysis and recommendations,
	\item \textbf{Internal specification} - description of the system architecture, code organization and sequence diagrams presenting interactions between major components during usage,
	\item \textbf{Verification and validation} - verification of the requirements, general system testing and presentation of bugs which were found and fixed,
	\item \textbf{Conclusions} - summary of the thesis and ideas for further development.
\end{itemize}
