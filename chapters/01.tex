\chapter{Introduction}
\label{chap:introduction}

The latest reports show that on average people spend over 6.5 hours daily using the internet \cite{bib:digital-2023}. Computers and the internet play a huge role in today's world. Huge amounts of data are transferred and processed by computers each second. Unfortunately, many services and applications are not secure. Consequences of vulnerabilities may be very different and vary from almost negligible to extremely severe. Some examples include service inaccessibility, data leaks and network takeovers. Sometimes seemingly harmless issues may be exploited in a series of attacks and cause much bigger problems. Of the over 8 billion of people on Earth \cite{bib:UN-8B-population} not everyone has good intentions. Malicious actors try to attack computer systems all the time. In 2017, it was estimated that over 2200 cyberattacks are preformed each day. While many of those attacks rely on human mistakes such as weak passwords, password reuse and susceptibility to phishing, attackers often try to exploit vulnerabilities in applications. The popularity of web applications makes them an especially valuable target for hackers. Internet applications frequently store lots of information about their users which can be sold on hacking forums and markets for profit, used in phishing attacks or sometimes even used for extortion if particularly sensitive data is stolen. Successful hacking attacks are also sources of password leaks. These are dangerous, because people tend to reuse the same passwords over and over and once a username-password pair is found it almost certainly will be used in so-called credential stuffing attacks to take over other accounts belonging to the same user without relying on vulnerabilities in those other platforms.

Like in many areas the best solution to these issues is prevention. Although code review and testing may help with problem detection and fixing, it would be much better if these issues were not created in the first place. According to Michał Sajdak the source of many vulnerabilities and one of the most important problems in security is the lack of knowledge about security issues \cite{bib:securitum-wstep}. A developer, especially an inexperienced one, may not be aware of potential issues with their code. Educating about vulnerabilities may help programmers look on their code from a different perspective, therefore making it more secure. Some programmers do not think about security, because they do not expect their applications to be ever attacked. They may sometimes consider hacking almost a form of magic, because they have never tried it themselves. It is important to raise awareness of such people about cybersecurity, exploitation approaches, vulnerability types and their prevention. Sometimes old tutorials are responsible for developers repeating vulnerable code patterns in their own projects, because they do not have better sources. A popular example of this problem in the world of web applications is usage of \texttt{Element.innerHTML}, which was used in older but still popular tutorials, despite often being the source of XSS vulnerabilities. To counteract this problem educational materials informing about these problems should be available on the internet and taught during programming classes and courses.

The aim of the thesis is creating a free and open-source system useful for teaching cybersecurity that could be used by anyone. The system is intended to utilize Capture The Flag (CTF) challenges for demonstrating the vulnerabilities. This is a popular challenge type requiring the user to exploit a security issue and find a hidden flag. This approach was chosen, because such challenges are usually very engaging and introduce gamification to the process of education, therefore making it more effective. Besides offering CTF challenges the system should be able to show descriptions of vulnerabilities, why those issues exist, what may be the possible consequences and how the dangerous code patterns can be avoided. Additionally, the platform is expected to offer quizzes, which can be used to verify also some knowledge of the theory. As an educational tool it is supposed to be easy to use for the students or other users and should support a system of hints.

The thesis consists of the planning and preparation of the project, creation of architecture capable of running challenges, designing and implementing user interface and a management panel. Additionally, existing solutions and ideas related to the problem were analysed before planning the system. The thesis consists of seven chapters:

\begin{enumerate}
	\item Chapter \ref{chap:introduction} gives an overview of the problem and describes the idea behind the thesis.
	\item In chapter \ref{chap:problem-analysis}, the problem and relevant literature are analysed. The chapter describes also the existing solutions and how they are different from the proposed one.
	\item The functional and non-functional requirements are formulated in chapter \ref{chap:req-and-tools}. Additionally, the chapter describes third-party tools and technologies used in the project as well as the design and planning process.
	\item Chapter \ref{chap:external-specification} includes details important for the users and administrators of the system. It describes the hardware and software requirements, installation and setup procedures and security considerations. This chapter includes also manuals for users and administrators.
	\item The system architecture and code organization are described in chapter \ref{chap:internal-specification}. This chapter presents also interactions between major components during usage of the platform in form of sequence diagrams.
	\item Chapter \ref{chap:verification-and-validation} contains details about verification of the requirements and general system testing. It ends with a presentation of bugs which were found and fixed during the testing procedure.
	\item Finally, chapter \ref{chap:conclusions} summarizes the thesis and lists ideas for further development.
\end{enumerate}
