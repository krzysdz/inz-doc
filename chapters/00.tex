\subsubsection*{Thesis title} \Title

\subsubsection*{Abstract}
The popularity of the internet and its broad usage in the contemporary world introduces new problems in the form of vulnerabilities. One of the best ways to prevent them is good and practical cybersecurity education of developers responsible for the said web services. The thesis aims to design and create an easy-to-use platform suitable for education about web security using capture the flag challenges. The project combines this popular type of tasks with multiple-choice quizzes and allows rich descriptions of vulnerability types. The solution uses Docker and nginx proxy for easy automated challenge management.

\subsubsection*{Keywords}
educational platform, cybersecurity training, CTF, web security, Docker

\subsubsection*{Tytuł pracy}
\begin{otherlanguage}{polish}
\TitleAlt
\end{otherlanguage}

\subsubsection*{Streszczenie}
\begin{otherlanguage}{polish}
Popularność internetu i jego powszechne użytkowanie we współczesnym świecie przyczynia się do powstawania nowego problemu w postaci luk bezpieczeństwa. Jednym z najlepszych sposobów zapobiegania temu problemowi jest dobra i praktyczna edukacja deweloperów aplikacji webowych z zakresu cyberbezpieczeństwa. Celem pracy jest zaprojektowanie i stworzenie łatwej w użytkowaniu platformy przeznaczonej do edukacji o bezpieczeństwie aplikacji internetowych wykorzystującej wyzwania capture the flag. Projekt łączy ten popularny rodzaj zadań z pytaniami wielokrotnego wyboru i pozwala na bogate opisy rodzajów podatności. Rozwiązanie wykorzystuje Dockera i proxy nginx do łatwego i zautomatyzowanego zarządzania wyzwaniami.
\end{otherlanguage}
\subsubsection*{Słowa kluczowe}
\begin{otherlanguage}{polish}
platforma edukacyjna, nauczanie cyberbezpieczeństwa, CTF, bezpieczeństwo aplikacji internetowych, Docker
\end{otherlanguage}

